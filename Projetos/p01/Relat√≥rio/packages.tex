% Codificação dos caracteres da entrada (use ISO no Windows e UTF no Linux)
%\usepackage[isolatin]{inputenc}  % arquivos LaTeX em ISO-8859-1
\usepackage[utf8]{inputenc}     % arquivos LaTeX em Unicode

% packages usados na construção deste documento
\usepackage [T1]{fontenc}        % caracteres acentuados corretos na saida
\usepackage {times}              % Fontes Times
\usepackage [portuguese]{babel}      % texto em portugues
\usepackage {indentfirst}        % indentar primeiro paragrafo
\usepackage {epsfig}             % inclusao de figuras em formato EPS
\usepackage[hidelinks]{hyperref}

% usar papel no formato A4 com margens 30,30,25,25 mm^M
\usepackage[a4paper,top=30mm,bottom=30mm,left=25mm,right=25mm]{geometry}

% relaxar o espaçamento entre caracteres
% \sloppy

% O espaçamento entre linhas deve ser 1.5
% \renewcommand{\baselinestretch}{1.5}

% indentação dos parágrafos é 15mm
\setlength{\parindent}{15mm}

% pacote sugerido para formata��o de c�digo-fonte
\usepackage{listings}
\lstset{language=c}
\lstset{inputencoding=latin1,extendedchars=true}
\lstset{basicstyle=\small,commentstyle=\textit,stringstyle=\ttfamily}
\lstset{showspaces=false,showtabs=false,showstringspaces=false}
\lstset{numbers=left,stepnumber=5,numberstyle=\tiny}
\lstset{columns=flexible,mathescape=true}
\lstset{frame=single}

% pacote sugerido para formatação de algoritmos
\usepackage{algorithm,algorithmic}
\floatname{algorithm}{Algoritmo}
\renewcommand{\algorithmiccomment}[1]{~~~// #1}
